\documentstyle[12pt,epsf,subfigure]{report}
%macro for Postscript figures the queasy way
\newcommand{\postscript}[2]
 {\setlength{\epsfxsize}{#2\hsize}
  \centerline{\epsfbox{#1}}}

\begin{document}

\title{Phaser Chess}
\author{By \em The Hungry Programmers}
\date{Version 0.5}
\maketitle

\noindent

\setcounter{chapter}{1}

\section*{Introduction}
\paragraph[]{}
Phaser Chess is an obvious steal from Compute Magazine's LaserChess.
The authors think that by changing some of the bitmaps a little and
adding a piece, they can avoid being jailed.

\paragraph[]{}
Each player is allowed 3 moves per turn.  The number of moves left are
shown in a small rectangle beneath the fire and pass controls.  Each
dot signifies a move left in the current turn.

\paragraph[]{}
To move a piece, select the piece with the left mouse button, and
click on a square near the piece.  No piece can jump over another
piece.  To move a piece 1 square takes 1 move.  To move it diagonally
takes 2 moves.  A piece can cross several squares per turn, or several
pieces can cross one square per turn.  The 3 moves can be distributed
among different pieces however the player wishes.  After a piece has
been moved, it is automatically de-selected.

\paragraph[]{}
To rotate a piece, select it, and press either the middle or the right
mouse buttons.  When it is at the desired rotation, de-select it with
the left mouse button.  Rotation takes a move, regardless of how far
the piece has been rotated.  It the piece is rotated back to its
original position before the piece is de-selected, a move is not
taken.  It a piece is selected and rotated, and the player clicks the
left mouse button on a square near the selected piece, the piece will
be moved to that square and rotated (if there are enough moves left in
that turn).


\newpage

\section*{Rules}
\paragraph[]{}

\begin{figure}[htb]
	\postscript{king.eps}{.12}
	\caption{The King}
	\label{king}
\end{figure}

The object of the game is to kill your enemy's king (figure
\ref{king}).  Some pieces kill by firing a phaser beam, some kill by
stomping, and some can't kill (directly).


\begin{figure}[htb]
	\postscript{phaser.eps}{.12}
	\caption{The Phaser}
	\label{phaser}
\end{figure}

\paragraph[]{}
The phaser (figure \ref{phaser}) can fire a beam which can destroy
pieces.  To fire the phaser, select the phaser with the left mouse
button, and press the 'Fire' button.  Firing a phaser take one move
from the turn.  A phaser can only be fired once per turn.  If the
player has another phaser, it can still be fired in the same turn.
Most pieces have some mirrored surfaces.  The mirrored surfaces show
up as lighter (brighter?)  edges.  If the phaser beam hits a
non-mirrored surface, the piece will explode.  If the phaser hits a
mirrored surface, the beam will bounce.  The phaser has no mirrored
surfaces, and will destroy itself if its beam is reflected back onto
itself (from any angle).


\begin{figure}[htb]
	\postscript{freezer.eps}{.12}
	\caption{The Freezer}
	\label{freezer}
\end{figure}

\paragraph[]{}
The freezer (figure \ref{freezer}) is similar to a phaser.  It fires a
beam which will react to mirrored surfaces in the same way as the
phaser's beam.  Firing a freezer takes a turn, and each freezer may
only be fired once per turn.  If the freezer's beam hits a
non-mirrored surface, the piece which has been hit will freeze and
turn blue.  While a piece is frozen, its owner is not able to move or
rotate it, and it's normally mirrored surfaces no longer reflect.
Because its mirrors no longer reflect, it may be more easily destroyed
by a phaser.  There is a 1 in 5 chance each frozen piece will thaw
every time a turn ends.  Freezers have 3 mirrored surfaces, and
are vulnerable to beams from all other directions.


\begin{figure}[htb]
	\postscript{pstomper.eps}{.12}
	\caption{The Stomper}
	\label{pstomper}
\end{figure}

\paragraph[]{}
Stompers (figure \ref{pstomper}) can stomp on other pieces to kill
them.  This is done by moving the stomper onto the square occupied by
the victim piece.  A stomper may only destroy one piece per turn.  A
stomper is able to stomp enemy pieces, as well as pieces which are the
same color as itself.  Most stompers have 3 mirrored surfaces.  If a
beam strikes any of the mirrored surfaces, the beam will reflect back
at 180 degrees from its original path.

\begin{figure}[htb]
	\postscript{fstomper.eps}{.12}
	\caption{The Fully Mirrored Stomper}
	\label{fstomper}
\end{figure}

Each player has one stomper which is mirrored on all 8 sides (figure
\ref{fstomper}).  It is not vulnerable from beam weapons, unless a
bomb is involved (see bombs, below).

\begin{figure}[htb]
	\postscript{mirror2.eps}{.12}
	\caption{A Mirror}
	\label{mirror}
\end{figure}

\paragraph[]{}
Mirrors (figure \ref{mirror}) cannot kill by themselves.  They can be
used to direct the beam of a phaser or a freezer at an enemy (or
friendly) piece.  If a beam strikes a mirror from strait on, the beam
will reflect at 180 degrees from its original path.  If a beam comes
from an angle (but still striking the mirrored surface of the mirror),
it will reflect at 90 degrees from its original path.  If a beam comes
from the side of the mirror, it will go past the mirror unaffected.
See figure \ref{mirror_ex}.

\begin{figure}[htb]
	\postscript{mirror_ex.eps}{.80}
	\caption{Beams and Mirrors}
	\label{mirror_ex}
\end{figure}


\begin{figure}[htb]
	\postscript{splitter.eps}{.12}
	\caption{A Splitter}
	\label{splitter}
\end{figure}

\paragraph[]{}
Splitters (figure \ref{splitter}) are special mirrors which can be
used to split a beam into 2 beams.  If a beam strikes a splitter from
its front, the beam will be split, and will go off in two directions
90 degrees from the beams original path.  If a beam strikes a splitter
from its side, the beam will be bent 90 degrees.  If the splitter is
hit from behind, it is vulnerable to the beam.  A beam from any other
direction will not be affected by the splitter.

\newpage

\begin{figure}[htb]
	\postscript{oneway.eps}{.12}
	\caption{A Oneway Mirror}
	\label{oneway}
\end{figure}

\paragraph[]{}

Oneway mirrors (figure \ref{oneway}) are mirrors which allow a beam to
pass through from the back, but reflect a beam which approaches from
the front.  If a beam strikes a oneway from any other direction, the
oneway is vulnerable.  Oneways are useful for protecting phasers and
freezers from their own beams.

\begin{figure}[htb]
	\postscript{teleporter.eps}{.12}
	\caption{A Teleporter}
	\label{teleporter}
\end{figure}

\paragraph[]{}
Teleporters (figure \ref{teleporter}) are similar to stompers, except
that instead of destroying whatever is stomped on, the victim is sent
to some random (and unoccupied) place on the board.  Each teleporter
may teleport one piece per turn.  There is a small chance that the
victim will land in one of the pits located vertically along the
center of the board (See pits, below).  If a beam hits a teleporter,
the beam will be bent to some random angle (possible strait back from
where it came).

\begin{figure}[htb]
	\postscript{gate.eps}{.12}
	\caption{A Gate}
	\label{gate}
\end{figure}

\paragraph[]{}
Gates (figure \ref{gate}) always come in pairs.  Gates cannot stomp or
kill in any way.  If a piece (other than the gates partner) moves onto
a gate, that piece will move out of the gate's parter in the same
direction (assuming there is space for it to land).  If a stomper
moves through a gate and lands on some other piece, the piece landed
on will be destroyed.  If a teleporter moves onto a gate, the gate
will be teleported (rather than the teleporter stepping out of the
gate's partner).  If one gate is destroyed, its partner is destroyed.
If one gate is frozen, its partner is frozen.  If a stomper steps on a
frozen gate, the gate and its partner are destroyed.  Gates are
vulnerable to beams from all directions.

\begin{figure}[htb]
	\postscript{bomb.eps}{.12}
	\caption{A Bomb}
	\label{bomb}
\end{figure}

\paragraph[]{}
Bombs (figure \ref{bomb}) cannot stomp and are vulnerable to beams
from all directions.  If a beam approaches from any of the four
directions which allow the beam to pass into the center of the bomb,
the bomb and all squares adjacent to the bomb will feel the effect of
the beam.  This effect cannot be defended from by mirrored surfaces.
If a beam hits a bomb from any of the four directions which do not
allow the beam to pass into the bomb, only the bomb itself will feel
the effects of the beam.

\begin{figure}[htb]
	\postscript{pit.eps}{.12}
	\caption{A Pit}
	\label{pit}
\end{figure}

\paragraph[]{}
Pits (figure \ref{bomb}) are not owned by either player -- they are
part of the board.  They appear and disappear at random.  They are
always positioned vertically along the center of the board.  If any
piece moves into a pit, or is teleported onto a pit, that piece will
be destroyed.  Pits do not affect beams in any way.  If phaserchess is
run with the {\tt --devil} option, the pits will appear as devils (figure
\ref{devil}).

\begin{figure}[htb]
	\postscript{devil.eps}{.12}
	\caption{A Devil Pit}
	\label{devil}
\end{figure}

\begin{figure}[htb]
	\postscript{centerpit.eps}{.12}
	\caption{The Center Teleporter}
	\label{centerpit}
\end{figure}

\paragraph[]{}
The center of the board will always be either a pit, or the center
teleporter (figure \ref{centerpit}).  It will toggle between these two
states randomly.  The center teleporter is not owned by either player.
If a piece steps or is teleported onto the center teleporter, it will
be teleported to some random place on the board.  There is a small
chance that the piece being teleported will land on an open pit and be
destroyed.

\end{document}
